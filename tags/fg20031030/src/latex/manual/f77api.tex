\subsection{Fortran API}

The Fortran API of LCIO is based on a set of wrapper functions to the 
C++ implementation using "cfrotran.h" to create the correct Fortran name.

The main idea is that we will use integers in Fortran to represent pointers to 
objects on the C++ side. There will be one wrapper function for every class method
of the implementation classes (namespaces IMPL and IO) plus two additional methods 
to create and delete the object respectively.
By using a unique naming convention the documentation of the C++ version of the API
can be utilized for the Fortran API as well.

\subsubsection{Naming convention}
The following naming convention is used for the fortran wrapper functions to the 
C++ implementation of LCIO:
\begin{itemize}
\item{all function names start with {\bf lc}}
\item{{\bf lc} is followed by a three letter acronym that uniquley identifies the 
corresponding C++ class, e.g. {\bf evt} for LCEvent.} See table \ref{tab_f77} for a complete 
listing.

\item{the function name ends on the full lowercased name of the class method, e.g. \\
 \verb$LCEventImpl::getRunNumber()$ becomes  {\bf lcevtgetrunnumber() } }
\item{the constructor and destructor of the class  end on {\bf create} and {\bf delete} 
respectively, eg. {\bf lcevtdelete()} }
\end{itemize}

\begin{table}
\begin{center}
\begin{tabular}{|c|c|}
\hline
\rule[-5mm]{0mm}{10mm} C++ class  &  f77 acronym   \\ \hline \hline

 LCRunHeaderImpl       & rhd \\ \hline
 LCEventImpl           & evt \\ \hline
 LCCollectionVec       & col \\ \hline
 MCParticleImpl        & mcp \\ \hline
 SimTrackerHitImpl     & sth \\ \hline
 SimCalorimeterHitImpl & sch \\ \hline
 CalorimeterHitImpl    & cah \\ \hline
 TPCHitImpl            & tph \\ \hline
 LCReader              & rdr \\ \hline
 LCWriter              & wrt \\ \hline

\end{tabular}
\end{center}
\caption{Three letter acronyms for f77 wrapper functions.}
\label{tab_f77}
\end{table}
