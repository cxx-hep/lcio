
{\large\bf The basic Fortran API to LCIO} \\

\begin{scriptsize}

{\bf Remarks:} \\

The return value of the functions and the meaning of arguments are either: \\
$*$ pointers denoted by a name beginning with the {\bf letter p} \\
$*$ character strings denoted by {\bf ...name} or {\bf string} \\
$*$ integers  denoted by {\bf status} or a variable name starting with {\bf i} or {\bf n}\\
$*$ double precision variables name starting with {\bf d} \\
$*$ reals {\bf else}
$*$ arrays are denoted by a name ending with {\bf v} \\


\begin{verbatim}

class LCReader:

create                 -> preader = lcrdrcreate()
delete                 -> status  = lcrdrdelete( preader )
open                   -> status  = lcrdropen( preader, filename )
close                  -> status  = lcrdrclose( preader )
readStream             -> status  = lcrdrreadstream( preader, nmax )
readNextRunHeader      -> pheader = lcreadnextrunheader( preader, iaccessmode )
readNextEvent          -> pevent  = lcrdrreadnextevent( preader, iaccessmode )
readEvent              -> pevent  = lcrdrreadevent( preader, irun, ievt )

RunEventProcessor (includes registerLCEventListener, registerLCRunListener)
                       -> status  = lcrdreventprocessor( filename )

class LCWriter:

create                 -> pwriter = lcwrtcreate()
delete                 -> status  = lcwrtdelete( pwriter )
open                   -> status  = lcwrtopen( pwriter, filename, imode )
close                  -> status  = lcwrtclose( pwriter )
writeRunHeader         -> status  = lcwrtwriterunheader( pwriter, pheader )
writeEvent             -> status  = lcwrtwriteevent( pwriter, pevent )


class LCRunHeader:

create                 -> pheader = lcrhdcreate()
delete                 -> status  = lcrhddelete( pheader )
setRunNumber           -> status  = lcrhdsetrunnumber( pheader )
setDetectorName        -> status  = lcrhdsetdetectorname( pheader , detname)
setDescription         -> status  = lcrhdsetdescription( pheader , descrstring )
addActiveSubdetector   -> status  = lcrhdaddactivesubdetector( pheader , sdname )

getRunNumber           -> irun    = lcrhgetrunnumber( pheader )
getDetectorName        -> detname = lcrhdgetdetectorname( pheader )
getDescription         -> descr   = lcrhdgetdescription( pheader )

getActiveSubdetectors by:
getActiveSubdetectors  -> psdvec  = lcrhdgetactivesubdetectors( pheader )
getNumberOfElements    -> nelem   = lcsvcgetlength( psdvec )                         (stl vector Interface)
getElementAt           -> sdname  = lcsvcgetstringat( psdvec , i ) (i=1,...,nelem)   (stl vector Interface)


class LCEvent:      

create                 -> pevent  = lcevtcreate()
delete                 -> status  = lcevtdelete( pevent )
setRunNumber           -> status  = lcevtsetrunnumber( pevent , irun)
setEventNumber         -> status  = lcevtseteventnumber( pevent , ievt)
setDetectorName        -> status  = lcevtsetdetectorname( pevent , detname)
setTimeStamp           -> status  = lcevtsettimestamp( pevent , itim)
addCollection          -> status  = lcevtaddcollection( pevent , pcol , colname)
removeCollection       -> status  = lcevtremovecollection( pevent , colname)
addRelation            -> status  = lcevtaddrelation( pevent , prelation, name)
removeRelation         -> status  = lcevtremovecollection( pevent , name)

getRunNumber           -> irun    = lcevtgetrunnumber( pevent )
getEventNumber         -> ievt    = lcevtgeteventnumber( pevent )
getDetectorName        -> detname = lcevtgetdetectorname( pevent )
getTimeStamp           -> itim    = lcevtgettimestamp( pevent )
getCollection          -> pcol    = lcevtgetcollection( pevent , colname)

getCollectionNames by:
getCollectionNames     -> pstv    = lcevtgetcollectionnames( pevent )
getNumberOfElements    -> nelem   = lcsvcgetlength( pstv )                            (stl vector Interface)
getElementAt           -> colname = lcsvcgetstringat( pstv , i ) (i=1,...,nelem)      (stl vector Interface)


class LCCollection:

create                 -> pcol    = lccolcreate( colname )
delete                 -> status  = lccoldelete( pcol )
addElement             -> status  = lccoladdelement( pcol , pobject)
removeElementAt        -> status  = lccolremoveelementat ( pcol , i )
setFlag                -> status  = lccolsetflag( pcol , iflag )

getTypeName            -> name    = lccolgettypename( pcol )
getNumberOfElements    -> number  = lccolgetnumberofelements( pcol )
getElementAt           -> pobject = lccolgetelementat( pcol , i )  (i=1,...,number)
getFlag                -> iflag   = lccolgetflag( pcol )


class LCRelation:

create                 -> prel    = lcrelcreate( fromtypename , totypename )
delete                 -> status  = lcreldelete( prel )
numberOfRelations      -> nrel    = lcrelnumberofrelations( prel, pobject )
getRelation            -> pgetrel = lcrelgetrelation( prel, pobject, i ) (i=1,...,nrel)
getWeight              -> weight  = lcrelgetweight( prel, pobject, i ) (i=1,...,nrel)

addRelation            -> status  = lcreladdrelation( prel, pobjectfrom, pobjectto, weight )
useCaching             -> status  = lcrelusecaching( prel, lval ) (lval = type logical)
setTypes               -> status  = lcrelsettypes( prel, fromtypename , totypename )

class SimTrackerHit:

create                 -> pthit   = lcsthcreate()
delete                 -> status  = lcsthdelete( pthit )
setCellID              -> status  = lcsthsetcellid( pthit , icellid )
setPosition            -> status  = lcsthsetposition( pthit , dposv )
setdEdx                -> status  = lcsthsetdedx( pthit , fdedx )
setTime                -> status  = lcsthsettime( pthit , ftime )
setMCParticle          -> status  = lcsthsetmcparticle( pthit , pmcp )

getCellID              -> icellid = lcsthgetcellid( pthit )
getPosition            -> dposv(i)= lcsthgetposition( pthit , i ) (i=1,2,3)
getdEdx                -> fdedx   = lcsthgetdedx( pthit )
getTime                -> ftime   = lcsthgettime( pthit )
getMCParticle          -> pmcp    = lcsthgetmcparticle( pthit )


class SimCalorimeterHit:

create                 -> pchit   = lcschcreate()
delete                 -> status  = lcschdelete( pchit )
setCellID0             -> status  = lcschsetcellid0( pchit , icellid0 )
setCellID1             -> status  = lcschsetcellid1( pchit , icellid1 )
setEnergy              -> status  = lcschsetenergy( pchit , energy )
setPosition            -> status  = lcschsetposition( pchit , posv )
addMCParticleContr..   -> status  = lcschaddmcparticlecontribution( pchit , pmcp , energy , time , ipdg )

getCellID0             -> icellid0= lcschgetcellid0( pchit )
getCellID1             -> icellid1= lcschgetcellid1( pchit )
getEnergy              -> energy  = lcschgetenergy( pchit )
getPosition            -> status  = lcschgetposition( pchit , posv )
getNMCParticles        -> number  = lcschgetnmcparticles( pchit )
getParticleCont        -> pmcp    = lcschgetparticlecont( pchit , i ) (i=1,...,number)
getEnergyCont          -> energy  = lcschgetenergycont( pchit , i )   (i=1,...,number)
getTimeCont            -> time    = lcschgettimecont( pchit , i )     (i=1,...,number)
getPDGCont             -> ipdg    = lcschgetpdgcont( pchit , i )      (i=1,...,number)


class CalorimeterHit:

create                 -> pchit   = lccahcreate()
delete                 -> status  = lccahdelete( pchit )
setCellID0             -> status  = lccahsetcellid0( pchit , icellid0 )
setCellID1             -> status  = lccahsetcellid1( pchit , icellid1 )
setEnergy              -> status  = lccahsetenergy( pchit , energy )
setPosition            -> status  = lccahsetposition( pchit , posv )

getCellID0             -> icellid0= lccahgetcellid0( pchit )
getCellID1             -> icellid1= lccahgetcellid1( pchit )
getEnergy              -> energy  = lccahgetenergy( pchit )
getPosition            -> status  = lccahgetposition( pchit , posv )


class TPCHit:

create                 -> pthit   = lctphcreate()
delete                 -> status  = lctphdelete( pthit )
setCellID              -> status  = lctphsetcellid( pthit , icellid )
setTime                -> status  = lctphsettime( pthit , time )
setCharge              -> status  = lctphsetcharge( pthit , charge )
setQuality             -> status  = lctphsetquality( pthit , iquality )
setRawDataWords        -> status  = lctphsetrawdata( pthit , irawv, nraw )

getCellID              -> icellid = lctphgetcellid( pthit )
getTime                -> time    = lctphgettime( pthit )
getCharge              -> charge  = lctphgcharge( pthit )
getQuality             -> iquality= lctphgetquality( pthit )
getRawDataWords        -> nraw    = lctphgetnrawdatawords( pthit )
% we should use the standard Fortran convention for indicees i.e. (i=1,...,nraw)
%                       -> irawv(i)= lctphgetrawdataword( pthit , i ) (i=1,...,nraw)
%FIXME: we need to hava consistent indices - either 0,n-1 as in C++ or 1,n !
                       -> irawv(i)= lctphgetrawdataword( pthit , i ) (i=0,...,nraw-1)


class TrackerHit:

create                 -> ptrhit  = lctrhcreate()
delete                 -> status  = lctrhdelete( ptrhit )
setPosition            -> status  = lctrhsetposition( ptrhit, dposv )
setCovMatrix           -> status  = lctrhsetcovmatrix( ptrhit, covmxv )
setdEdx                -> status  = lctrhsetdedx( ptrhit, dedx )
setTime                -> status  = lctrhsettime( ptrhit, time )
setTPCHit              -> status  = lctrhsettpchit( ptrhit, pthit )

id                     -> id      = lctrhid( ptrhit )
getPosition            -> status  = lctrhgetposition( ptrhit, dposv )
getCovMatrix           -> status  = lctrhsetcovmatrix( ptrhit, covmxv )
getdEdx                -> dedx    = lctrhgetdedx( ptrhit )
getTime                -> time    = lctrhgettime( ptrhit )
getType                -> string  = lctrhgettype( ptrhit )
getRawDataHit          -> prawdh  = lctrhgetrawdatahit( ptrhit )


class Track:

create                 -> ptrk    = lctrkcreate()
delete                 -> status  = lctrkdelete( ptrk )
setTypeBit             -> status  = lctrksettypebit( ptrk, ibit)
setOmega               -> status  = lctrksetomega ( ptrk, omega )
setTanLambda           -> status  = lctrksettanlambda( ptrk, tanlambda )
setPhi                 -> status  = lctrksetphi( ptrk, phi )
setD0                  -> status  = lctrksetd0( ptrk, d0 )
setZ0                  -> status  = lctrksetz0( ptrk, z0 )
setCovMatrix           -> status  = lctrksetcovmatrix( ptrk, covmxv )
setReferencePoint      -> status  = lctrksetreferencepoint( ptrk, refpointv )
setIsReferencePointPCA -> status  = lctrksetisreferencepointpca( ptrk, irefp )
setChi2                -> status  = lctrksetchi2( ptrk, chi2 )
setNdf                 -> status  = lctrksetndf( ptrk, ndf )
setdEdx                -> status  = lctrksetdedx( ptrk, dedx )
setdEdxError           -> status  = lctrksetdedxerror( ptrk, dedxerr )
setRadiusOfInnermostHit-> status  = lctrksetradiusofinnermosthit( ptrk, radius)
addTrack               -> status  = lctrkaddtrack( ptrk, ptrack )
addHit                 -> status  = lctrkaddhit( ptrk, phit )
subdetectorHitNumbers  -> status  = lctrksetsubdetectorhitnumbers( ptrk, intv, nintv)

id                     -> id      = lctrkid( ptrk )
getType                -> itype   = lctrkgettype( ptrk )
testType               -> ittype  = lctrktesttype( ptrk, ibit )
getOmega               -> omega   = lctrkgetomega( ptrk )
getTanLambda           -> tanlam  = lctrkgettanlambda( ptrk )
getPhi                 -> phi     = lctrkgetphi( ptrk )
getD0                  -> d0      = lctrkgetd0( ptrk )
getZ0                  -> z0      = lctrkgetz0( ptrk )
getCovMatrix           -> status  = lctrkgetcovmatrix( ptrk, covmxv )
getReferencePoint      -> status  = lctrkgetreferencepoint( ptrk, refpointv )
isReferencePointPCA    -> irefp   = lctrkisreferencepointpca( ptrk )
getChi2                -> chi2    = lctrkgetchi2( ptrk )
getNdf                 -> ndf     = lctrkgetndf( ptrk )
getdEdx                -> dedx    = lctrkgetdedx( ptrk )
getdEdxError           -> dedxerr = lctrkgetdedxerror( ptrk )
getRadiusOfInnermostHit-> radius  = lctrkgetradiusofinnermosthit( ptrk )
subdetectorHitNumbers  -> status  = lctrkgetsubdetectorhitnumbers( ptrk, intv, nintv)
getTracks              -> ptrackv = lctrkgettracks( ptrk )
getTrackerHits         -> ptrhitv = lctrkgettrackerhits( ptrk )


class Cluster:

create                 -> pclu    = lcclucreate()
delete                 -> status  = lccludelete( pclu )
setTypeBit             -> status  = lcclusettypebit( pclu, ibit)
setEnergy              -> status  = lcclusetenergy( pclu, energy )
setPosition            -> status  = lcclusetposition( pclu, posv )
setPositionError       -> status  = lcclusetpositionerror( pclu, poserrv )
setTheta               -> status  = lcclusettheta( pclu, theta )
setPhi                 -> status  = lcclusetphi( pclu, phi )
setDirectionError      -> status  = lcclusetdirectionerror( pclu, direrrv )
setShape               -> status  = lcclusetshape( pclu, shapev )
setEMWeight            -> status  = lcclusetemweight( pclu, emweight )
setHADWeight           -> status  = lcclusethadweight( pclu, hadweight )
setMuonWeight          -> status  = lcclusetmuonweight( pclu, muonweight )
addCluster             -> status  = lccluaddcluster( pclu, pcluadd )
addHit                 -> status  = lccluaddhit( pclu, pcalhit, weight )

id                     -> id      = lccluid( pclu )
getType                -> itype   = lcclugettype( pclu )
testType               -> itype   = lcclutesttype( pclu )
getEnergy              -> energy  = lcclugetenergy( pclu )
getPosition            -> status  = lcclugetposition( pclu, posv )
getPositionError       -> status  = lcclugetpositionerror( pclu, poserrv )
getTheta               -> status  = lcclugettheta( pclu, theta )
getPhi                 -> theta   = lcclugetphi( pclu, phi )
getDirectionError      -> phi     = lcclugetdirectionerror( pclu, direrr )
getShape               -> status  = lcclugetshape( pclu, shapev )
getParticleType        -> status  = lcclugetparticletype( pclu, weightsv) (weightsv(1,2,3))
getHitContributions    -> status  = lcclugethitcontributions( pclu, energiesv, nenergies)
getSubdetectorEnergies -> pfloatv = lcclugetsubdetectorenergies( pclu )
getCalorimeterHits     -> pcalhv  = lcclugetcalorimeterhits( pclu )
getClusters            -> pcluget = lcclugetclusters( pclu )



class ReconstructedParticle:

create                 -> prcp    = lcrcpcreate()
delete                 -> status  = lcrcpdelete( prcp )
setType                -> status  = lcrcpsettype( prcp, itype )
setPrimary             -> status  = lcrcpsetprimary( prcp, lprimary )  (lprimary = type logical)
setMomentum            -> status  = lcrcpsetmomentum( prcp, xmomv )
setEnergy              -> status  = lcrcpsetenergy( prcp, energy )
setCovMatrix           -> status  = lcrcpsetcovmatrix( prcp, covmxv )
setMass                -> status  = lcrcpsetmass( prcp, xmass )
setCharge              -> status  = lcrcpsetcharge( prcp, charge ) ;
setReferencePoint      -> status  = lcrcpsetreferencepoint( prcp, refpointv ) ;
addParticleID          -> status  = lcrcpaddparticleid( prcp, pid ) ;
addParticle            -> status  = lcrcpaddparticle( prcp, pparticle, weigth ) ;
addCluster             -> status  = lcrcpaddcluster( prcp, pclus, weigth ) ;
addTrack               -> status  = lcrcpaddtrack( prcp, ptrack, weigth ) ;
addMCParticle          -> status  = lcrcpaddmcparticle( prcp, pmcp, weigth ) ;

getType                -> itype   = lcrcpgettype( prcp )
isPrimary              -> lprim   = lcrcpisprimary( prcp ) (lprim = type logical)
getMomentum            -> status  = lcrcpgetmomentum( prcp, xmomv )
getEnergy              -> energy  = lcrcpgetenergy( prcp )
getCovMatrix           -> status  = lctrkgetcovmatrix( prcp, covmxv )
getMass                -> xmass   = lcrcpgetmass( prcp )
getCharge              -> charge  = lcrcpgetcharge( prcp )
getReferencePoint      -> status  = lcrcpgetreferencepoint( prcp, refpointv )
getParticleIDs         -> pids    = lcrcpgetparticleids( prcp )
getParticles           -> ppartv  = lcrcpgetparticles( prcp )
getParticleWeights     -> status  = lcrcpgetparticleweights( prcp, weightsv, nweights )
getClusters            -> pclusv  = lcrcpgetclusters( prcp )
getClusterWeights      -> status  = lcrcpgetparticleweights( prcp, weightsv, nweights )
getTracks              -> ptrkv   = lcrcpgettracks( prcp )
getTrackWeights        -> status  = lcrcpgettrackweights( prcp, weightsv, nweights )
getMCParticles         -> pmcpv   = lcrcpgetmcparticles( prcp )
getMCParticleWeights   -> status  = lcrcpgetmcparticleweights( prcp, weightsv, nweights )


class ParticleID:

create                 -> ppid    = lcpidcreate()
delete                 -> status  = lcpiddelete( ppid )
setTypeID              -> status  = lcpidsettypeid( ppid, idtype )
setProbability         -> status  = lcpidsetprobability( ppid, prob )
setIdentifier          -> status  = lcpidsetidentifier( ppid, idname )
addParameter           -> status  = lcpidaddparameter( ppid, param )

id                     -> id      = lcpidid( ppid )
getTypeID              -> idtype  = lcpidgettypeid( ppid )
getProbability         -> prob    = lcpidgetprobability( ppid )
getIdentifier          -> idname  = lcpidgetidentifier( ppid )
getParameters          -> status  = lcpidgetparameters( ppid, paramv, nparam)


class MCParticle:

create                 -> pmcp    = lcmcpcreate()
delete                 -> status  = lcmcpdelete( pmcp )
addParent              -> status  = lcmcpaddparent( pmcp , pmcpp )
setPDG                 -> status  = lcmcpsetpdg( pmcp , ipdg )
setGeneratorStatus     -> status  = lcmcpsetgeneratorstatus( pmcp , istatus )
setSimulatorStatus     -> status  = lcmcpsetsimulatorstatus( pmcp , istatus )
setVertex              -> status  = lcmcpsetvertex( pmcp , dvtxv )
setEndpoint            -> status  = lcmcpsetendpoint( pmcp , dvtxv )
setMomentum            -> status  = lcmcpsetmomentum( pmcp , momv )
setMass                -> status  = lcmcpsetmass( pmcp , mass )
setCharge              -> status  = lcmcpsetcharge( pmcp , charge )

getNumberOfParents     -> number  = lcmcpgetnumberofparents( pmcp )
getParent              -> pmcpp   = lcmcpgetparent( pmcp , i )
getNumberOfDaughters   -> number  = lcmcpgetnumberofdaughters( pmcp )
getDaughter            -> pmcpd   = lcmcpgetdaughter( pmcp , i )  (i=1,...,number)
getPDG                 -> ipdg    = lcmcpgetpdg( pmcp )
getGeneratorStatus     -> istatg  = lcmcpgetgeneratorstatus( pmcp )
getSimulatorStatus     -> istats  = lcmcpgetsimulatorstatus( pmcp )
getVertex              -> status  = lcmcpgetvertex( pmcp , dvtxv )
getEndpoint            -> status  = lcmcpgetendpoint( pmcp , dvtxv )
getMomentum            -> status  = lcmcpgetmomentum( pmcp , momv )
getEnergy              -> energy  = lcmcpgetenergy( pmcp )
getMass                -> mass    = lcmcpgetmass( pmcp )
getCharge              -> charge  = lcmcpgetcharge( pmcp )


utility (stl vector Interface):

getLengthofIntVector   -> nelem   = lcivcgetlength( pintvec )
getIntAt               -> status  = lcivcgetintat( pintvec , i ) (i=1,...,nelem)
getLengthofFloatVector -> nelem   = lcfvcgetlength( pfloatvec )
getFloatAt             -> float   = lcfvcgetfloatat( pfloatvec , i ) (i=1,...,nelem)
getLengthofStringVector-> nelem   = lcsvcgetlength( pstrvec )
getStringAt            -> string  = lcsvcgetstringat( pstrvec , i ) (i=1,...,nelem)

\end{verbatim}

\end{scriptsize}



\newpage
{\large\bf The extended Fortran API to LCIO} \\

\begin{scriptsize}

{\bf Remarks:} \\

The return value of the functions and the meaning of arguments are either: \\
$*$ pointers denoted by a name beginning with the {\bf letter p} \\
$*$ character strings denoted by {\bf ...name} or {\bf string} \\
$*$ integers  denoted by {\bf status} or a variable name starting with {\bf i} or {\bf n} \\
$*$ double precision variables denoted by a name starting with {\bf d} \\
$*$ arrays denoted by a name ending with {\bf v} \\
$*$ reals {\bf else} \\
If arguments of the type {\bf array of character strings} are used the last
argument has to be the length of a character string in the array.   \\
Integers starting with n are also used to give the length of an array 
(input/output argument for get... functions, input: dimension of the array, output: number of values stored).


\begin{verbatim}

for class LCReader:

lcrdropenchain         -> status = lcrdropenchain(preader, filenamesv, nfiles, len(filenamesv(1)))

for class LCRunHeader:

writeRunHeader         -> status  = lcwriterunheader( pwriter, irun, detname, descrname, sdnamev, nsdn, len(sdnamev(1)) )
readNextRunHeader      -> pheader = lcreadnextrunheader( preader, irun, detname, descrname, sdnamev, nsdn, len(sdnamev(1)) )

for class LCEvent:

setEventHeader         -> status  = lcseteventheader ( pevent, irun, ievt, itim, detname )
getEventHeader         -> status  = lcgeteventheader ( pevent, irun, ievt, itim, detname )
dumpEvent              -> status  = lcdumpevent ( pevent )
dumpEventDetailed      -> status  = lcdumpeventdetailed ( pevent )

for class SimTrackerHit:

addSimTrackerHit       -> status  = lcaddsimtrackerhit( pcolhitt, icellid, dposv, fdedx, ftime, pmcp )
getSimTrackerHit       -> status  = lcgetsimtrackerhit( pcolhitt, i, icellid, dposv, fdedx, ftime, pmcp )

for class SimCalorimeterHit:

addSimCaloHit          -> phit    = lcaddsimcalohit( pcolhitc, icellid0, icellid1, energy, posv )
addSimCaloHitMCont     -> status  = lcschaddmcparticlecontribution( phit, pmcp, energy, time, ipdg )  (from basic f77 API)
getSimCaloHit          -> phit    = lcgetsimcalohit( pcolhitc, i, icellid0, icellid1, energy, posv )
getSimCaloHitMCont     -> status  = lcgetsimcalohitmccont( phit, i, pmcp, energy, time, ipdg )

for class MCParticle:

getMCParticleData      -> status  = lcgetmcparticledata ( pmcp, ipdg, igstat, isstat, dvtxv, momv, mass, charge, ndaughters )

for class HEPEVT (Fortran interface  to COMMON /HEPEVT/ ):

toHepEvt               -> status  = lcio2hepevt( pevent )
fromHepEvt             -> status  = hepevt2lcio( pevent )

for stl vector Interface:

createIntVector        -> pvec    = lcintvectorcreate( intv, nint )
createFloatVector      -> pvec    = lcfloatvectorcreate( floatv, nfloat )
createStringVector     -> pvec    = lcstringvectorcreate( stringsv, nstrings, len(stringsv(1)) )

getIntVector           -> status  = lcgetintvector( pvec , intv, nintv )
getFloatVector         -> status  = lcgetfloatvector( pvec , floatv , nfloatv )
getStringVector        -> status  = lcgetstringvector( pvec , stringv , nstringv, len(stringv(1)) )

\end{verbatim}

\end{scriptsize}


