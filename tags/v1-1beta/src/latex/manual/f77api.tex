\subsection{The Fortran Interface} \label{f77api}

The Fortran API of LCIO is based on a set of wrapper functions to the 
C++ implementation using "cfrotran.h" to create the correct Fortran name.

The main idea is that we use integers in Fortran to represent pointers to 
objects on the C++ side. There will be one wrapper function for every class method
of the implementation classes (namespaces IMPL and IO) plus two additional methods 
to create and delete the object respectively. All functions that operate on objects, i.e.
all functions except for the creation functions (constructors) need as a first argument 
the integer pointer to the particluar object.

By using a unique naming convention the documentation of the C++ version of the API
can be utilized for the Fortran API as well.

Moreover example code written in C++ can be translated into Fortran basically line by 
line as far as LCIO is concerned -- of course language specific control structures 
will have to be different. So even if you are only intereseted in using LCIO from Fortran 
it is probably a good idea to read the above sections on Java and C++ as well to get some 
insight into the general structure of LCIO.

\subsubsection{Naming convention}
The following naming convention is used for the fortran wrapper functions to the 
C++ implementation of LCIO:
\begin{itemize}

\item{all function names start with {\bf lc}}

\item{{\bf lc} is followed by a three letter acronym that uniquley identifies the 
corresponding C++ class, e.g. {\bf evt} for LCEvent.} See table \ref{tab_f77} for a complete 
listing.

\item{the function name ends on the full lowercased name of the class method, e.g. \\
 \verb$LCEventImpl::getRunNumber()$ becomes  {\bf lcevtgetrunnumber() } }

\item{the constructor and destructor of the class  end on {\bf create} and {\bf delete} 
respectively, eg. {\bf lcevtdelete()} }

\item{All constants defined in \verb$Event::LCIO$ are defined as constants in Fortran with the 
same name prepended by {\bf 'LCIO\_'}, e.g. the type name for MCParticles defined in C++ in
\verb$LCIO::MCPARTICLE$ is defined in a character constant in Fortran named
{\bf LCIO\_MCPARTICLE} (see the Fortran include file {\bf lciof77apiext.inc}).}

\end{itemize}

Two additional methods are defined for each case to handle string, int and float vectors in Fortran:\\

{\bf lcsvcgetlength()}, {\bf lcsvcgetstringat()}  \\
{\bf lcivcgetlength()}, {\bf lcivcgetintat()}  \\
{\bf lcfvcgetlength()} and {\bf lcfvcgetfloatat()} \\

For strings the Fortran CHARACTER* declaration has to be large enough, otherwise the original
string is truncated.


\begin{table}
\begin{center}
\begin{tabular}{|c|c|}
\hline
\rule[-5mm]{0mm}{10mm} C++ class  &  f77 acronym   \\ \hline \hline

 LCRunHeaderImpl       & rhd \\ \hline
 LCEventImpl           & evt \\ \hline
 LCCollectionVec       & col \\ \hline
 MCParticleImpl        & mcp \\ \hline
 SimTrackerHitImpl     & sth \\ \hline
 SimCalorimeterHitImpl & sch \\ \hline
 CalorimeterHitImpl    & cah \\ \hline
 TPCHitImpl            & tph \\ \hline
 LCReader              & rdr \\ \hline
 LCWriter              & wrt \\ \hline

\end{tabular}
\end{center}
\caption{Three letter acronyms for f77 wrapper functions.}
\label{tab_f77}
\end{table}


\subsubsection{Reading and writing LCIO files}
Examples for reading and writing LCIO files can be found in:
\begin{verbatim}
  src/f77
    simjob.F
    anajob.F
\end{verbatim}
To build these examples, do:
\begin{verbatim}
   gmake -C src/f77
\end{verbatim}

The complete interface is declared in the include file \verb#$LCIO/src/f77/lciof77api.inc#. 
%$
A simple example for reading an LCIO file with Fortran:

\begin{verbatim}
 ...
#include "lciof77api.inc"
#include "lciof77apiext.inc"

      PTRTYPE reader, event, runhdr
      integer status, irun, ievent

      reader = lcrdrcreate()
      status = lcrdropen( reader, 'simjob.slcio' )
      
      if( status.eq.LCIO_ERROR) then
         goto 99
      endif

      do 
         event = lcrdrreadnextevent( reader )
         if( event.eq.0 ) goto 11 
         
         status = lcdumpevent( event )
      enddo
 11   continue
 ...
\end{verbatim}
The function \verb$lcdumpevent(event)$ is part of the extended interface described in \ref{f77ext}.
Note that all functions that operate on existing objects have as a first argument the integer pointer 
to this particluar object.
All functions that do not return a pointer to an object do return a status word instead that can be 
compared to \verb$LCIO_ERROR$ (\verb$LCIO_SUCCESS$).

And accordingly writing an LCIO file from Fortran:

\begin{verbatim}
 ...
      writer = lcwrtcreate()
      status = lcwrtopen( writer, filename , LCIO_WRITE_NEW )

 ... 
      do iev = 1,nev
          
          event = lcevtcreate()

          status = lcevtsetrunnumber( event, irun ) 
          status = lcevtseteventnumber( event,  iev ) 
 
          schcol = lccolcreate( LCIO_SIMCALORIMETERHIT )

          do k=1,nhit

             hit = lcschcreate() 
             status = lcschsetcellid0( hit, 1234 ) 
             status = lcschsetenergy( hit, energy )

             status = lccoladdelement( schcol, hit ) 
          enddo

          status = lcwrtwriteevent( writer , event )
          
c------- need to delete the event as we created it
          status = lcevtdelete( event )

      enddo

      status = lcevtaddcollection(event,schcol ,'MyCalHits') 

      status = lcwrtclose( writer ) 
 ...
\end{verbatim}

Note that as in the C++ case we have to delete the event if we created it as described in \ref{cppmem}. \\
A Summary of all functions in the basic Fortran API is given in the first part of {\bf Appendix~\ref{ftn_summary} }.



\subsubsection{Extension of the Base Fortran API} \label{f77ext}
An additional set of Fortran functions is provided for user convenience. These are higher level functions
that usually allow to access several attributes of data objects with one function call.
These functions are declared in \verb#$LCIO/src/f77/lciof77apiext.inc#,
%$
and are listed in the second part of {\bf Appendix~\ref{ftn_summary} }.

Most Fortran programs for simulation use the \verb$hepevt$ common block. 
Conversion functions from the LCIO MCParticle collection to the \verb$hepevt$ common block and vice versa 
are also provided in the extended interface. See the example in \ref{rwpythia}.
